\chapter{Jednoduché rozhranie}

\label{kap:prakticke} % id kapitoly pre prikaz ref

V tejto kapitole navrhneme, implementujeme a otestujeme jednoduché rozhranie so schémou zabezpečenia využívajúcou jednotlivé tokeny popísané v kapitole \ref{kap:typy} a teoreticky porovnané v kapitole \ref{kap:teoreticke}. Nepriehľadný token budeme, rovnako ako v prechádzajúcej kapitole, chápať ako náhodný reťazec podpísaný pomocou hešovania s kľúčom. Cieľom implementácie je porovnať rýchlosť spracovania požiadaviek využívajúcich rôzne tokeny na autorizáciu a jednoduchosť práce s knižnicami implementujúcimi jednotlivé tokeny vo vybranom programovacom jazyku. Implementácia môže zároveň slúžiť ako základ pre implementovanie jednoduchej schémy zabezpečenia a jej integrovanie s vybranými knižnicami. Zdrojový kód je dostupný na GitHube\footnote{Zdrojový kód je dostupný na adrese https://github.com/jitka1997/bachelor-thesis/tree/main/simpleAPI}.

\section{Použité technológie}

Jednoduché rozhranie sme implementovali v jazyku JavaScript, konkrétne pomocou prostredia Node.js. Použili sme aplikačný rámec Express.js, ktorý umožňuje ľahko implementovať jednoduchý server.

Na vytváranie a validáciu jednotlivých tokenov sme použili knižnice pre JavaScript z~tabuľky \ref{tab:implementacie}, ktoré sme použili pri porovnaní popularity tokenov. Pre Biscuits neexistuje knižnica pre JavaScript, preto sme Biscuits v tejto kapitole neporovnávali. Prácu s~nepriehľadným tokenom sme implementovali sami. Informácie spojené s nepriehľadným tokenom si ukladáme v databáze. Využili sme jednoduchú databázu sqlite.

Klienta vykonávajúceho požiadavky na rozhranie sme tiež naprogramovali v Node.js.

\section{Popis fungovania rozhrania}

Vytvorili sme rozhranie s dvoma koncovými bodmi \textit{signin} a \textit{welcome}. Prvý z nich slúži na získanie tokenu klientom a druhý na vykonanie požiadavky, ktorá bude úspešná len v~prípade, že bude obsahovať platný token, ktorý rozhranie úspešne validuje. Na získanie tokenu sa klient musí úspešne autentifikovať pomocou základnej HTTP autentifikácie, teda zaslaním base64url \cite{base64_rfc} zakódovaného mena a hesla v autentifikačnej hlavičke požiadavky.

Rozhranie overí korektnosť prihlasovacích údajov a ak sú správne vygeneruje token. V reálnom systéme by boli prihlasovacie údaje uložené v databáze, kde by ich rozhranie overilo. My pre jednoduchosť databázu používateľov simulujeme pomocou načítania objektu s používateľmi zo súboru usersDB. V tomto objekte sú používateľské mená a~heslá uložené ako dvojice kľúč a hodnota.

Po úspešnej autentifikácii klienta vygeneruje rozhranie token a vráti ho klientovi. Klient následne môže token použiť na vykonanie požiadavky na koncový bod \textit{welcome}. Rozhranie overí platnosť tokenu a ak je platný, vykoná požiadavku, teda vráti klientovi uvítaciu správu obsahujúcu jeho prihlasovacie meno, získané z tokenu. Ak token nebol zaslaný alebo zlyhá jeho validácia, rozhranie vráti klientovi chybový kód.

\section{Obsah a generovanie tokenu}
\label{sec:obsah}

Obsahovo vytvárame token nesúci štyri informácie overované pri autorizácii. Konkrétne čas vypršania platnosti tokenu, prihlasovacie meno používateľa, identifikátor vydavateľa tokenu a identifikátor prijímateľa tokenu. Čas vypršania platnosti tokenu je 5~minút od jeho vydania, identifikátory vydavateľa a prijímateľa sú pevne stanovené hodnoty a prihlasovacie meno je získané z prihlasovacích údajov klienta.

Informácie sme do obsahu tokenu obsiahli, aby sme demonštrovali využitie údajov z tokenu na autorizáciu požiadavky, prípadne identifikáciu používateľa. Do konkrétneho tokenu sme vložili informácie spôsobom, ktorý nám umožnila špecifikácia daného tokenu a použitá knižnica implementujúca token. Konkrétne:


\begin{itemize}
    \item Nepriehľadný token -- do databázy sme vložili riadok obsahujúci samotný token ako identifikátor a ostatné informácie ako hodnoty jednotlivých stĺpcov.
    \item JWT a PASETO -- meno používateľa sme vložili do tela tokenu ako oprávnenie \textit{username} a ostatné informácie ako štandardné oprávnenia.
    \item Fernet -- všetky informácie sme vložili ako serializovaný JSON objekt do tela tokenu. Fernet síce obsahuje samostatnú časovú pečiatku, no vybraná knižnica nepodporuje jej použitie na validáciu časovej platnosti tokenu.
    \item Branca -- čas vypršania platnosti tokenu sme zaznamenali ako časovú pečiatku vytvorenia tokenu a následne pri jeho validácii sme určili ako stará môže táto časová pečiatka byť. Ostatné informácie sme vložili do tela tokenu ako serializovaný JSON objekt.
    \item Macaroons -- všetky informácie sme do tokenu vložili ako pravidlá prvej strany.
\end{itemize}

JWT a PASETO, aj vybrané knižnice, ktoré ich implementujú, ponúkajú viacero funkcií na podpísanie a prípadné šifrovanie tokenu. Ostatné tokeny však používajú na podpisovanie jedinú funkciu, vždy variant hešovania s kľúčom. Fernet a Branca navyše šifrujú telo tokenu. Preto sme pre objektívne porovnanie tokenov zvolili pri JWT a~PASETO, čo najpodobnejšie funkcie na podpisovanie a taktiež šifrujeme telo tokenu. Kryptografické funkcie použité na podpisovanie a šifrovanie jednotlivých tokenov sú uvedené v tabuľke \ref{tab:funkcie}.

\begin{table}
  \begin{center}
    \caption{Kryptografické funkcie na podpisovanie a šifrovanie tokenov}
    \label{tab:funkcie} % create a label for the table, after caption

    \resizebox{\columnwidth}{!}{%
    \begin{tabular}{lcccccc}
      \hline
      Proces & Nepriehľadný & JWT & PASETO & Fernet & Branca & Macaroons \\
      \hline
      Šifrovanie & $\varoslash$ & AES-128-CBC & AES-256-CTR & AES-128-CBC & XChaCha20 & $\varoslash$ \\
      Podpisovanie & HMAC-SHA256 & HMAC-SHA256 & HMAC-SHA384 & HMAC-SHA256 & Poly1305 & HMAC-SHA256 \\
      \hline
    \end{tabular}%
    }
  \end{center}
\end{table}

\section{Práca s knižnicami}

Pre každý token sme pomocou danej knižnice implementovali 2 funkcie -- \textit{createToken(username)} a \textit{verifyToken(token)}. Funkcia \textit{createToken} vráti nový token s obsahom a formátom popísaným v podkapitole \ref{sec:obsah}. Funkcia \textit{verifyToken} validuje podpis tokenu. Následne dešifruje telo tokenu (ak bolo zašifrované) a validuje jeho obsah. Ako kľuč pre kryptografické funkcie sme používali konštantný náhodne vygenerovaný reťazec s~potrebným počtom bitov.

Jednoduchosť práce s knižnicami implementujúcimi tokeny porovnávame na základe podpory jednoduchého generovania tokenu a štandardných validácií tela tokenu.
Všetky knižnice podporujú jednoduché podpísanie a prípadné šifrovanie tokenu. Takisto všetky knižnice podporujú validáciu podpisu tokenu. 

Pre JWT a PASETO nám knižnice ponúkli viacero štandardných oprávnení, ktoré sme mohli použiť pri vytváraní obsahu tokenu. Potom pri validácii tokenu stačilo uviesť požadované hodnoty týchto oprávnení a knižničné volanie ich validovalo. Fernet nepodporuje nijaké štandardné validovanie obsahu tokenu ani na úrovni špecifikácie, teda ani knižnica, ktorá ho implementuje, ho neponúkala. Overovanie formátu a obsahu tela tokenu sme teda implementovali sami. Branca podobne ako Fernet na úrovni špecifikácie neponúka štandardnú validáciu tela tokenu. No vybraná knižnica umožňuje validovať aspoň časovú platnosť tokenu pomocou časovej pečiatky v tokene a časového limitu zadaného ako argument pri dešifrovaní tela. Ani špecifikácia Macaroons nepodporuje žiadnu štandardnú validáciu pravidiel. Vybraná knižnica dovoľuje jednoducho pridávať pravidlá prvej strany tým, že riadi réžiu okolo postupného generovania podpisu tokenu. Neexistujú však žiadne štandardné pravidlá a preto sme ich validáciu implementovali sami.

Na základe týchto pozorovaní sme porovnali jednoduchosť práce s rôznymi tokenmi, rovnakým spôsobom ako vlastnosti porovnané v tabuľke \ref{tab:porovnanie} pomocou symbolov \CIRCLE, \LEFTcircle, \Circle ~a $\varoslash$. Výsledky sú uvedené v tabuľke \ref{tab:api_porovnanie}.

\section{Meranie rýchlosti}

Pre porovnanie rýchlosti spracovania požiadavky sme implementovali jednoduchého klienta vykonávajúceho požiadavky na rozhranie. Klient sa najprv úspešne autentifikuje voči rozhraniu, ktoré mu vráti platný prístupový token. Následne klient vytvorí požiadavku na rozhranie s týmto tokenom.

Klient teda vykonáva jednu požiadavku bez tokenu, na ktorej vykonanie musí rozhranie vygenerovať nový token a jednu požiadavku s platným tokenom, na ktorej vykonanie musí rozhranie validovať token. Teda v princípe meriame čas generovania a validácie tokenu. Na meranie času vytvorenia tokenu meria klient čas od odoslania požiadavky na autentifikáciu po vrátenie tokenu rozhraním. Na meranie času validácie tokenu meria klient čas od odoslania požiadavky na rozhranie po vrátenie odpovede rozhraním.

Pre presnejšie meranie času vykoná klient 10000 požiadaviek na autentifikáciu a~10000 požiadaviek s platným tokenom a zaznamená priemer a medián nameraných časov. Pri meraní času sme paralelne pustili 5 klientov, ktorých zaznamenané priemery a mediány sme spriemerovali. Výsledky sú uvedené v tabuľke \ref{tab:api_porovnanie}.

\begin{table}
  \begin{center}
    \caption{Namerané rýchlosti spracovania požiadavky}
    \label{tab:api_porovnanie} % create a label for the table, after caption

    \resizebox{\columnwidth}{!}{%
    \begin{tabular}{lcccccc}
      \hline
      Vlastnosť & Nepriehľadný & JWT & PASETO & Fernet & Branca & Macaroons \\
      \hline
      Jednoduchosť práce & $\varoslash$ & \CIRCLE & \CIRCLE & \Circle & \Circle & \LEFTcircle \\
      Generovanie priemer [ms] & 17.556 & 2.442 & 2.241 & 1.577 & 2.089 & 2.430 \\
      Generovanie medián [ms] & 17.087 & 2.160 & 2.093 & 1.457 & 1.898 & 2.266 \\
      Validácia priemer [ms] & 21.230 & 1.846 & 1.803 & 1.199 & 1.336 & 2.183 \\
      Validácia medián [ms] & 23.216 & 1.743 & 1.706 & 1.152 & 1.274 & 2.079 \\
      \hline
    \end{tabular}%
    }
  \end{center}
\end{table}

\section{Vyhodnotenie nameraných výsledkov}
\label{sec:vyhodnotenie}

Keď budeme hovoriť o rýchlosti tokenu, myslíme tým rýchlosť generovania a validácie daného tokenu. 

Vidíme, že nepriehľadný token je výrazne pomalší ako ostatné tokeny. Konkrétne je v priemere v prípade generovania 7-11 krát pomalší a v prípade validácie 10-16 krát pomalší. Toto je spôsobené tým, že rozhranie si pri generovaní nového tokenu musí tento token uložiť do databázy spolu s autorizačnými údajmi a pri validácii musí token spolu s údajmi vyhľadať v databáze.

Ďalej si všimnime, že JWT a PASETO majú skoro identické časy pri všetkých meraných hodnotách. Tento výsledok je pri použití podobných kryptografických funkcií očakávaný, lebo ide o veľmi podobné tokeny. Fernet a Branca sú dva najrýchlejšie tokeny, čo je pravdepodobne spôsobené jednoduchosťou ich formátu. Branca je však priemerne 1.3 krát pomalší pri generovaní a 1.1 krát pomalší pri validácii ako Fernet. Tento rozdiel môžeme prisúdiť použitiu rôznych kryptografických funkcií alebo inej implementácii.

Prekvapivým výsledkom je, že Macaroons je najpomalší pri generovaní aj validácii, pretože okrem nepriehľadného tokenu ide o jediný token, ktorého telo sa nešifruje. Tento výsledok je možné vysvetliť tým, že Macaroons je najkomplexnejší token, formátom aj procesom generovania. Pri generovaní sa aplikuje podpisová funkcia raz pri vytvorení nového tokenu a následne toľko krát koľko pravidiel do tokenu pridávame, v našom prípade štyri krát. Podobne pri validácii tokenu sa musí vypočítať podpis tokenu postupnou aplikáciu podpisovej funkcie na každé pravidlo v tele tokenu. Macaroons bol vytvorený pre komplexné systémy pre jeho flexibilitu spočívajúcu v možnosti pridávať pravidlá ľubovoľnou entitou a elegantného riešenia zapojenia tretích strán do autorizácie pomocou pravidiel tretích strán. Preto je pre jednoduchý systém, ako je nami implementované rozhranie, Macaroons nevhodný.

Porovnaním mediánu a priemeru hodnôt vidíme, že medián je takmer vždy menší oproti priemeru. Jedinou výnimkou je validácia nepriehľadného tokenu. Menší medián ako priemer v nameraných hodnotách značí, že medzi nameranými hodnotami bolo viac vysokých extrémov. Rozhranie aj klienta sme spúšťali lokálne z jedného počítaču, teda sa mohlo stať, že výkonnosť rozhrania ovplyvnili iné bežiace procesy na počítači, čo spôsobilo aj výkyvy v nameraných hodnotách.