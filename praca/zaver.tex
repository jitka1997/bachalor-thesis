\chapter*{Záver}  % chapter* je necislovana kapitola
\addcontentsline{toc}{chapter}{Záver} % rucne pridanie do obsahu
\markboth{Záver}{Záver} % vyriesenie hlaviciek

V práci sme sa venovali problematike autentifikácie a autorizácie so zameraním na tokeny. Práca sa dá pomyselne rozdeliť na dva logické celky. V prvej časti sme sa venovali teoretickým poznatkom, ktoré sú potrebné pre pochopenie problematiky. 

Konkrétne v kapitole \ref{kap:vyuzitie} sme uviedli rôzne známe prístupy k architektúre schémy zabezpečenia. Jednotlivé prístupy sme stručne charakterizovali a uviedli ich hlavné výhody a nevýhody. Okrem toho sme, pre prehľad v problematike tokenov, popísali známe typy a formáty tokenov. Ďalej sme sa venovali iba schéme zabezpečenia využívajúcej tokeny.

V kapitole \ref{kap:typy} sme predstavili konkrétne tokeny, ktoré sme ďalej v práci porovnávali. Venovali sme sa najmä charakteristikám a parametrom tokenov, ktoré sú nevyhnutné pre analýzu ich vlastností a porovnávanie s inými tokenmi.

V druhej časti práce sme sa venovali analýze, teoretickému a potom aj praktickému porovnaniu tokenov. V kapitole \ref{kap:teoreticke} sme porovnali tokeny z pohľadu bezpečnosti, flexibility a popularity. Výsledkom tohto porovnania je prehľadová tabuľka \ref{tab:porovnanie}. V kapitole \ref{kap:prakticke} sme navrhli a implementovali jednoduché rozhranie, na ktorom sme demonštrovali použitie jednotlivých tokenov v schéme zabezpečenia. Taktiež sme implementovali jednoduchého klienta, ktorý vykonával požiadavky na rozhranie a meral čas ich vykonania. Merali sme dva typy požiadaviek. Pri prvom type muselo rozhranie vygenerovať token a pri druhom validovať token. Namerané hodnoty a ich interpretácia pomocou rôznych grafov je v sekcii \ref{sec:vyhodnotenie}. Z meraní sme zistili, že najrýchlejšie tokeny sú Fernet a PASETO, no rozdiel v rýchlostiach tokenov JWT, PASETO, Branca a Fernet je takmer zanedbateľný. Pri validácii tokenu je tento rozdiel najviac 0,13 ms na jednu požiadavku. Naopak výrazne najpomalší bol nepriehľadný token z~dôvodu potreby použitia databázy pri generovaní aj validácii tokenu. Okrem nepriehľadného tokenu bol najpomalší Macaroons. Ide však o veľmi flexibilný token s komplexnou štruktúrou, preto nie je vhodný na použitie v jednoduchom rozhraní, ktoré sme implementovali.


Hlavným cieľom práce bolo porovnať rôzne tokeny. Ako výsledok tohto porovnania uvádzame tabuľku \ref{tab:vysledok}, ktorá je spojením tabuliek \ref{tab:porovnanie} a nameraných priemerných rýchlostí tokenov. Podrobnejšia interpretácia uvedených hodnôt je v kapitolách \ref{kap:teoreticke} a \ref{kap:prakticke}.

\begin{table}[H]
    \begin{center}
      \caption{Porovnanie tokenov}
      \label{tab:vysledok} % create a label for the table, after caption
  
      \resizebox{\columnwidth}{!}{%
      \begin{tabular}{lccccccc}
        \hline
        Vlastnosť & Nepriehľadný & JWT & PASETO & Fernet & Branca & Macaroons & Biscuits\\
        \hline
        Počet kryptografických funkcií & 1 & 30 & 6 & 1 & 1 & 1 & 1\\
        Určenie podpisového alg. z tokenu & $\varoslash$ & \CIRCLE & \CIRCLE & \Circle & \Circle & \Circle & \Circle \\
        Náchylnosť na útok pomýlením algortimu & $\varoslash$ & \CIRCLE & \LEFTcircle & \Circle & \Circle & \Circle & \Circle \\
        Riešenie problému odvolania & \CIRCLE & \Circle & \Circle & \Circle & \Circle & \LEFTcircle & \LEFTcircle \\
        Náchylnosť na útok opakovaním & $\varoslash$ & \CIRCLE & \CIRCLE & \CIRCLE & \CIRCLE & \LEFTcircle & \LEFTcircle \\
        Ochrana dôvernosti & $\varoslash$ & \LEFTcircle & \LEFTcircle & \CIRCLE & \CIRCLE & \Circle & \Circle \\
        Overenie autenticity a integrity hocikým & \Circle & \LEFTcircle & \CIRCLE & \CIRCLE & \CIRCLE & \CIRCLE & \CIRCLE \\
        zoslabenie tokenu hocikým & $\varoslash$ & $\varoslash$ & $\varoslash$ & $\varoslash$ & $\varoslash$ & \CIRCLE & \CIRCLE \\
        Autorizačná schéma v tokene & $\varoslash$ & \LEFTcircle & \LEFTcircle & \LEFTcircle & \LEFTcircle & \CIRCLE & \CIRCLE \\
        Bezstavová validácia & $\varoslash$ & \CIRCLE & \CIRCLE & \CIRCLE & \CIRCLE & \CIRCLE & \CIRCLE \\
        Štandardná validácia & $\varoslash$ & \CIRCLE & \CIRCLE & \Circle & \Circle & \LEFTcircle & \CIRCLE \\
        Popularita & $\varoslash$ & \CIRCLE & \LEFTcircle & \LEFTcircle & \Circle & \LEFTcircle & \Circle \\
        \hline
        Jednoduchosť práce & $\varoslash$ & \CIRCLE & \CIRCLE & \Circle & \Circle & \LEFTcircle & $\varoslash$ \\
        Generovanie priemer [ms] & 366,82 & 24,34 & 11,93 & 19,31 & 39,92 & 45,21 & $\varoslash$ \\
        Validácia priemer [ms] & 650,09 & 18,27 & 15,31 & 9,53 & 22,31 & 49,82 & $\varoslash$ \\
        \hline
      \end{tabular}%
      }
    \end{center}
  \end{table}

Pri porovnaní rýchlosti tokenov na testovacom API si však uvedomujeme, že porovnávame iba konkrétne implementácie daných tokenov. Implementácie sme vybrali na základe popularity a udržiavanosti knižníc, ktoré ich sprostredkúvajú. Analyzovať konkrétne implementácie tokenov presahuje rozsah tejto práci. Jednou z možností ako nadviazať na túto prácu je preto porovnanie viacerých implementácií každého tokenu a analýza a porovnanie týchto implementácií.

Okrem toho je v budúcnosti je možné nadviazať na našu prácu buď porovnaním ďalších tokenov, ktoré vzniknú v budúcnosti alebo porovnaním tokenov, ktoré sme v práci nespomenuli (napríklad token na základe certifikátov verejných kľúčov a Crypto auth token (CAT), oba vymyslené \cite{fb_tokens} a používané spoločnosťou Meta). Ďalej je možné nadviazať na našu prácu rozšírením rozhrania o ďalšie funkcie, ktoré by boli zaujímavé z pohľadu porovnania tokenov, napríklad pridať tretie strany do procesu autorizácie požiadavky.
