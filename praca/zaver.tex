\chapter*{Záver}  % chapter* je necislovana kapitola
\addcontentsline{toc}{chapter}{Záver} % rucne pridanie do obsahu
\markboth{Záver}{Záver} % vyriesenie hlaviciek

V práci sme sa venovali problematike autentifikácie a autorizácie v schéme zabezpečenia so zameraním na tokeny a ich porovnanie. Práca sa dá pomyselne rozdeliť na dva logické celky. V prvej časti sme sa venovali teoretickým poznatkom, ktoré sú potrebné pre pochopenie problematiky. 

Konkrétne v kapitole \ref{kap:vyuzitie} sme uviedli rôzne známe prístupy k architektúre schémy zabezpečenia. Jednotlivé prístupy sme stručne charakterizovali a uviedli ich hlavné výhody a nevýhody. Okrem toho sme, pre prehľad v tematike tokenov, popísali známe typy a formáty tokenov. Ďalej sme sa venovali iba schéme zabezpečenia využívajúcej tokeny.

V nasledujúcej kapitole \ref{kap:typy} sme predstavili konkrétne tokeny, ktoré sme ďalej v práci porovnávali. Venovali sme sa najmä takým charakteristikám a parametrom tokenov, ktoré sú nevyhnutné pre analýzu ich vlastností a porovnávanie s inými tokenmi.

V druhej časti práce sme sa venovali analýze a najprv teoretickému a potom aj praktickému porovnaniu tokenov. V kapitole \ref{kap:teoreticke} sme porovnali tokeny z pohľadu bezpečnosti, flexibility a popularity. Výsledkom tohto porovnania je prehľadová tabuľka \ref{tab:porovnanie}. V kapitole \ref{kap:prakticke} sme navrhli a implementovali jednoduché rozhranie, na ktorom sme demonštrovali použitie jednotlivých tokenov v schéme zabezpečenia. Taktiež sme implementovali jednoduchého klienta, ktorý vykonával požiadavky na rozhranie a meral čas ich vykonania. Klient posielal také požiadavky, na ktorých vykonanie, muselo rozhranie vygenerovať token alebo validovať token. Namerané hodnoty sú uvedené v tabuľke \ref{tab:api_porovnanie} a ich interpretácia v sekcii \ref{sec:vyhodnotenie}.

Hlavným cieľom práce bolo porovnať rôzne tokeny. Ako výsledok tohto porovnania uvádzame tabuľku \ref{tab:vysledok}, ktorá je spojením tabuliek \ref{tab:porovnanie} a \ref{tab:api_porovnanie}. Pre interpretáciu nameraných hodnôt je nutné si prečítať kapitoly \ref{kap:teoreticke} a \ref{kap:prakticke}.

\begin{table}
    \begin{center}
      \caption{Porovnanie tokenov}
      \label{tab:vysledok} % create a label for the table, after caption
  
      \resizebox{\columnwidth}{!}{%
      \begin{tabular}{lccccccc}
        \hline
        Vlastnosť & Nepriehľadný & JWT & PASETO & Fernet & Branca & Macaroons & Biscuits\\
        \hline
        Počet kryptografických funkcií & 1 & 30 & 6 & 1 & 1 & 1 & 1\\
        Určenie podpisového alg. z tokenu & $\varoslash$ & \CIRCLE & \CIRCLE & \Circle & \Circle & \Circle & \Circle \\
        Náchylnosť na útok pomýlením algortimu & $\varoslash$ & \CIRCLE & \LEFTcircle & \Circle & \Circle & \Circle & \Circle \\
        Riešenie problému odvolania & \CIRCLE & \Circle & \Circle & \Circle & \Circle & \LEFTcircle & \LEFTcircle \\
        Náchylnosť na útok opakovaním & $\varoslash$ & \CIRCLE & \CIRCLE & \CIRCLE & \CIRCLE & \LEFTcircle & \LEFTcircle \\
        Ochrana dôvernosti & $\varoslash$ & \LEFTcircle & \LEFTcircle & \CIRCLE & \CIRCLE & \Circle & \Circle \\
        Overenie autenticity a integrity hocikým & \Circle & \LEFTcircle & \CIRCLE & \CIRCLE & \CIRCLE & \CIRCLE & \CIRCLE \\
        zoslabenie tokenu hocikým & $\varoslash$ & $\varoslash$ & $\varoslash$ & $\varoslash$ & $\varoslash$ & \CIRCLE & \CIRCLE \\
        Autorizačná schéma v tokene & $\varoslash$ & \LEFTcircle & \LEFTcircle & \LEFTcircle & \LEFTcircle & \CIRCLE & \CIRCLE \\
        Bezstavová validácia & $\varoslash$ & \CIRCLE & \CIRCLE & \CIRCLE & \CIRCLE & \CIRCLE & \CIRCLE \\
        Štandardná validácia & $\varoslash$ & \CIRCLE & \CIRCLE & \Circle & \Circle & \LEFTcircle & \CIRCLE \\
        Popularita & $\varoslash$ & \CIRCLE & \LEFTcircle & \LEFTcircle & \Circle & \LEFTcircle & \Circle \\
        \hline
        Jednoduchosť práce & $\varoslash$ & \CIRCLE & \CIRCLE & \Circle & \Circle & \LEFTcircle & $\varoslash$ \\
        Generovanie priemer [ms] & 17.556 & 2.442 & 2.241 & 1.577 & 2.089 & 2.430 & $\varoslash$ \\
        Generovanie medián [ms] & 17.087 & 2.160 & 2.093 & 1.457 & 1.898 & 2.266 & $\varoslash$ \\
        Validácia priemer [ms] & 21.230 & 1.846 & 1.803 & 1.199 & 1.336 & 2.183 & $\varoslash$ \\
        Validácia medián [ms] & 23.216 & 1.743 & 1.706 & 1.152 & 1.274 & 2.079 & $\varoslash$ \\
        \hline
      \end{tabular}%
      }
    \end{center}
  \end{table}

V budúcnosti je možné nadviazať na našu prácu buď porovnaním ďalších tokenov, ktoré vzniknú v budúcnosti alebo porovnaním tokenov, ktoré sme v práci nespomenuli. Ďalej je možné nadviazať na našu prácu rozšírením rozhrania o ďalšie funkcie, ktoré by boli zaujímavé z pohľadu porovnania tokenov, napríklad pridať tretie strany do procesu autorizácie požiadavky.