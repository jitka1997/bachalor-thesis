\chapter*{Úvod} % chapter* je necislovana kapitola
\addcontentsline{toc}{chapter}{Úvod} % rucne pridanie do obsahu
\markboth{Úvod}{Úvod} % vyriesenie hlaviciek

Pri nie plne verejných aplikačných rozhraniach (API) je potrebné navrhnúť spôsob autentifikácie a autorizácie. Tento spôsob a jeho riadenie sa nazýva schéma zabezpečenia API.

Cieľom tejto práce je porovnanie konkrétnych API tokenov. API token slúži na autorizáciu požiadaviek na API. Naša práca môže slúžiť ako základ pri vytváraní schémy zabezpečenia v rámci vývoja API, konkrétne pri výbere adekvátneho API tokenu ku štruktúre a funkcionalite API. V práci sa zameriame na vybrané API tokeny, konkrétne: Nepriehľadný token, JWT, PASETO, Fernet, Braca, Macaroons a Biscuits. 

API tokeny sú úzko spojené s problematikou autentifikácie a autorizácie vo všeobecnosti. Preto sa zo začiatku práce venujeme aj tejto problematike, kde predstavujeme základné schémy zabezpečenia softvérových systémov.

Pre komplexné porovnanie vybraných API tokenov je nutné pochopenie ich fungovania. Teda pochopenie štruktúry API tokenu, spôsobu jeho vytvorenia a overenia. Preto v práci detailne popisujeme špecifikácie jednotlivých API tokenov.

API tokeny nejakým spôsobom autorizujú požiadavky na API. Taktiež môžu prenášať aj citlivé informácie o používateľoch. Preto je zaujímavé sa venovať bezpečnosti API tokenov. Moderné systémy často pozostávajú z veľa komponentov, ktoré medzi sebou komunikujú. Okrem iného môžu medzi sebou komunikovať aj pre účely autorizácie požiadaviek. Preto ak používajú v svojej architektúre API tokeny, je dôležité aby boli schémy týchto API tokenov schopné pracovať nad viacerými službami. Každý používaný softvér sa vyvíja, preto je pre dobré fungovanie akéhokoľvek softvéru je dôležité, aby bol aktívne udržiavaný a schopný reagovať na vývoj ostatných softvérov. Aby sa autorovi softvéru oplatilo ho udržiavať, musí byť tento softvér používaný a teda populárny.

Všetky tieto pozorovania sa dajú zhrnúť do troch vlastností -- bezpečnosť, flexibilita a popularita. Tieto vlastnosti tvoria dôležité aspekty, ktoré je potrebné zohľadniť pri výbere API tokenu. Preto im sa v práci pri porovnaní API tokenov venujeme a porovnávame ich medzi rôznymi API tokenmi.

Významným parametrom API je jeho výkonnosť. Teda koľko požiadaviek vie spracovať za určitý časový interval. Generovanie a validácia API tokenu sa spája aj s výpočtom rôznych kryptografických funkcií, ktoré sú často náročné na výpočet. Preto je pri porovnaní API tokenu zaujímavé zohľadniť aj rýchlosť jeho generovania a validácie. Z tohto dôvodu v práci navrhneme a implementujeme jednoduché API, na ktorom porovnáme rýchlosť generovania a validácie vybraných API tokenov.
