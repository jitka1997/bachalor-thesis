\chapter{Teoretické porovnanie API tokenov}

\label{kap:teoreticke} % id kapitoly pre prikaz ref

V tejto kapitole porovnáme rôzne parametre konkrétnych tokenov podľa informácií získaných z ich dokumentácií a iným zdrojov. Tieto informácie sme zhrnuli v kapitole \ref{kap:typy}. Pri jednotlivých parametroch vysvetlíme ich význam a teda aj dôležitosť pri porovnávaní tokenov. Porovnávať budeme všetky tokeny popísané v kapitole \ref{kap:typy} a nepriehľadný token popísaný v podkapitole \ref{sec:opaque}. Nepriehľadný token je formát tokenu, nie konkrétny token. Pre jednoduchosť budeme v tejto kapitole pod pojmom nepriehľadný token myslieť náhodný reťazec s podpisom, ktorý vznikol pomocou asymetrického širovania. 

Kapitola je štruktúrovaná podľa porovnávaných vlastností a jej výsledkom je tabuľka \ref{tab:porovnanie} zhrňujúca závery porovnania.

\section{Bezpečnosť}

V rámci porovnávania bezpečnosti tokenov nebudeme detailne rozoberať bezpečnosť jednotlivých kryptografických funkcií. Detaily ohľadom týchto funkcií je možné nájsť v ich citovaných dokumentáciach. Všetky tokeny ponúkajú možnosť použiť kryptografické funkcie, ktoré sú všeobecne považované za bezpečné.

Zameriame sa na porovnanie kryptografických primitív a z nich vyplývajúcich bezpečnostných kvalít a na náchylnosti na zraniteľnosti vyplývajúcej zo špecifikácie tokenu.

\subsection{Kryptografické primitíva}

Pri tokenoch rozoznávame tri kryptografické primitíva a to asymetrické šifrovanie vo forme elektronického podpisu, symetrické šifrovanie a hešovanie s kľúčom. Výstupom hešovania s kľúčom je hešovaný autentifikačný kód. 

Symetrické šiforvanie sa vrámci nami porovnávaných tokenov využíva na šifrovanie obsahu tokenu a teda na ochranu dôvernosti informácií uložených v tokene. Elektronický podpis a hešovanie zaručujú ochranu autenticity a integrity tokenu. Rozdiel v použití elektronického podpisu a hešovania je v tom, že v prípade elektronického podpisu ide o asymetrické šifrovanie, teda  podpis vie overiť ľubovoľná entita, ktorá pozná verejný kľúč tvoriaci dvojicu so súkromných kľúčom, ktorým bol token podpísaný. Takýto verejný kľúč je zväčša verejne dostupný a vie ho získať ľubovoľná entita. V prípade hešovania ide o symetrickú kryptografiu, pravosť hešovaného autentifikačného tokenu vie overiť len entita, ktorá pozná tajný kľúč, ktorým bol token zahešovaný, čo je často len entita, ktorá token vytvorila.

Výhodou elektronického podpisu teda je, že autenticitu a integritu tokenu môže overiť ľubovoľná entita. Výhodou hešovania je, že je rýchlejšie. 

V prípade JWT si môžeme vybrať, či budeme používať elektronický podpis alebo hešovanie s kľúčom a pomocou nastavenia oprávnenia \textit{alg} v hlavičke na požadovanú hodnotu. Všetky možnosti hodnôt oprávanenia \textit{alg} definuje  JWA \cite{jwa_rfc}. Štandard ponúka aj možnosť \textit{alg=none}, v tomto prípade nezaručuje JWT žiadne bezpečnostné kvality a je to jedna zo známych zraniteľností \cite{jwt_vul} JWT. Ak služba akceptuje aj JWT s \textit{alg=none} ako platné tokeny, útočník jednoducho zamení hodnotu \textit{alg='hocičo'} na \textit{none}, odstráni podpis z tokenu a môže ľubovoľne zmeniť token, napríklad si zvýši autorizačné práva. Bezpečné implementácie JWT, by nemali tokeny s \textit{alg=none} považovať za platné.

PASETO využíva v prípade lokálneho využitia hešovanie a v prípade verejného využitia elektronický podpis. Fernet, Branca a Macaroon využívajú hešovanie s kľúčom a Biscuit využíva elektronický podpis. Nepriehľadný token sme pre potreby tejto kapitoly definovali s použitím elektronického podpisu.

Symetrické šifrovanie a z neho vyplývajúcu ochranu dôvernosti umožňujú tokeny JWT, konkrétne vo forme JWE, PASETO s lokálnym využitím, Fernet a Branca. Biscuits a Macaroons neposkytujú žiadnu ochranu dôvernosti. Nepriehľadný token tiež neposkytuje ochranu dôvernosti, no z definície nenesie žiadnu informáciu, teda v jeho prípade nie je dôvernosť čoho chrániť.